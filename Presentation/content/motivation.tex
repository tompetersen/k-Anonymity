\section{Motivation \& Abgrenzung}

\begin{frame}{Motivation}

Einführendes Beispiel, dass das Setting und in die Begriffe einführt

	%Erstellt von script/generate_table.py - bei Bedarf anpassen
	\begin{tabular}{|l|c|c|r|l|}
	\hline \textbf{Identifier} & \multicolumn{3}{c|}{\textbf{Nicht-sensibel}} & \textbf{Sensibel} \\ 
	\hline \textbf{Name} & \textbf{Geschlecht} & \textbf{PLZ} & \textbf{Geburtsdatum} & \textbf{Erkrankung} \\
	\hline \hline Mia Schulz & w & 21989 & 20.5.1944 & Osteoporose \\ 
	\hline Elias Wagner & m & 21727 & 25.8.1983 & Gicht \\ 
	\hline Hanna Weber & w & 20817 & 28.3.1953 & Osteoporose \\ 
	\hline Leon Schulz & m & 21220 & 28.10.1994 & Bronchitis \\ 
	\hline Sofia Koch & w & 20270 & 21.1.1965 & Gicht \\ 
	\hline Leon Schmidt & m & 20188 & 5.5.1958 & Hepatitis \\ 
	\hline Hanna Schäfer & w & 21462 & 11.2.1999 & Epilepsie \\ 
	\hline Elias Schneider & m & 20388 & 3.8.1971 & Multiple Sklerose \\ 
	\hline Mia Fischer & w & 21896 & 14.12.1999 & Diabetes \\ 
	\hline Ben Meyer & m & 21024 & 8.1.1982 & Diabetes \\ 
	\hline 
	\end{tabular}
\end{frame}

\begin{frame}{Anonym?}
Sweeney - Beispiel \cite{sweeney_k_anonymity}

---

- Group Insurance Commission veröffentlichte Patientendaten in anonymer Form

- Cambridge, Massachusetts voter registration list mit den öffentlichen Informationen der GIC abgeglichen

- Beispielsweise konnte Massachusetts governor William Weld eindeutig identifiziert werden.

-> \cite{sweeney_demographics}, \cite{golle_demographics} Studien über die Eindeutigkeit von demographischen Faktoren in der U.S.-Bevölkerung \textbf{MAL REINSCHAUEN}

\end{frame}

\begin{frame}{Abgrenzung}

Abgrenzung zu anderen Konzepten (statistische Datenbanken, Authentifikation, ...)

\end{frame}
