\section{k-Anonymität}

\begin{frame}{k-Anonymität}
Notizen

identifier, quasi-identifier, sensitive attributes

k-anonymity
\end{frame}

\begin{frame}{Begriffe}
	\begin{description}
	\item[Explicit identifier] Attribut, das ein Individuum (nahezu) eindeutig identifiziert. Bsp: Name, Adresse, Steuernummer, ...
	
	\item[Sensitive attribute] Attribut, dessen Wert für ein Individuum in einer Datenmenge nicht herausgefunden werden darf.
	
	\item[Quasi identifier] Attributmenge, die ein Individuum in Kombination identifizieren kann. \textit{Formal in \cite{sweeney_k_anonymity} p. 7} \\
	\textit{auch \cite{machanavajjhala_l_diversity} p. 3}: Eine Menge nicht-sensibler Attribute \(\{A_i, \dots, A_j\}\) einer Tabelle, deren Attribute mit einer externen Datenquelle verknüpft werden können, um mindestens ein Individuum der Gesamtmenge eindeutig zu identifizieren.
	\end{description}
\end{frame}

\begin{frame}{k-Anonymität}
	\textbf{Informell}: Eine Tabelle (Datensatz?) erfüllt \(k\)-Anonymität, wenn jede Zeile (jeder Eintrag) ununterscheidbar von \(k-1\) anderen Zeilen im Bezug auf jede "'quasi identifier"'-Menge ist.

	\vspace{0.5cm}

	\textbf{Formal}: Sei \(T(A_1, \dots , A_n)\) eine Tabelle und \(Q_T=\{A_i, \dots, A_j\}\) der zugehörige quasi identifier. \(T\) erfüllt \(k\)-Anonymität genau dann, wenn jede Belegung von Werten in \(T[Q_T]\) mindestens \(k\) mal auftritt, wobei \(T[Q_T]\) die duplikatenerhaltende Projektion von \(T\) auf die Attribute des quasi identifiers beschreibt.
\end{frame}

\subsection{Generalisierung}

\begin{frame}{Generalisierung}
	\textbf{s. Samarati, Sweeney Kapitel 3}
	
	domain, ground domain, generalization (partial ordering on domains)
	
	generalized table
\end{frame}

\subsection{Unterdrückung}

\begin{frame}{Suppression}
	...
	
	in combination with generalization
\end{frame}

\section{Schwächen der k-Anonymität}

\begin{frame}{Schwächen der k-Anonymität}

\begin{description}
	\item[Unsorted matching attack] Veröffentlichung mehrerer \(k-\)anonymer Tabellen mit derselben Sortierung ausgehend von einer nicht-öffentlichen Tabelle. \cite{sweeney_k_anonymity} p.10

	\item[Complementary release attack] Veröffentlichung mehrerer \(k-\)anonymer Tabellen unterschiedlicher Generalisierung, die zusammengeführt die \(k-\)Anonymität verletzen. \cite{sweeney_k_anonymity} p.11

	\item[Temporal attack] Dynamische Tabellen können \(k-\)Anonymität verletzen. \cite{sweeney_k_anonymity} p.12

	\item[Homogeneity attack] Gleichheit der sensitive attributes einer Gruppe, die sich in den Werten des quasi identifiers gleicht, leakt das sensitive attribute eines Individuums. \cite{machanavajjhala_l_diversity} p. 2

	\item[Background knowledge attack] Nutzen von Hintergrundwissen, um mit hoher Wahrscheinlichkeit auf den Wert des sensitive attributes eines Individuumsin einer Gruppe zu schließen. \cite{machanavajjhala_l_diversity} p.2/4
\end{description}

\end{frame}