\section{l-Diversität}

\begin{frame}{l-Diversität - Prinzip}

\begin{description}
	\item[Prinzip] Eine Tabelle erfüllt \(l\)-Diversität, wenn in jedem \(k\)-anonymen Block mindestens \(l\) verschiedene Werte für das sensitive Attribut vorkommen.
    
    \vspace{1.0cm}
    
    \item[Bsp.-Tabelle]
    \small
	\begin{tabular}{|c|c|c|}
	\hline \textbf{G.jahr} & \textbf{PLZ} & \textbf{Erkrankung} \\
	\hline
     1970 & 21*** & Hepatitis X \\ 
	 1970 & 21*** & Hepatitis Y \\ 
	 1970 & 21*** & Hepatitis Z \\
	\hline 
     1975 & 21*** & Hepatitis X \\
     1975 & 21*** & Hepatitis X \\ 
	 1975 & 21*** & Hepatitis X \\ 
	\hline 
	\end{tabular}
	\vspace{0.3cm}

	\(k = 3, l = 1\)
\end{description}

\end{frame}


\begin{frame}{l-Diversität - Definitionen}
	\begin{block}{Entropie basierte l-Diversität \tiny \cite{machanavajjhala_l_diversity}}
        Eine Tabelle ist \textbf{\(l\)-divers}, wenn für jeden $q*$-Block die folgende Ungleichung erfüllt wird:
    	\[- \sum\limits_{s \in S} p_{(q^*,s)} \log(p_{(q^*,s)}) \geq log(l)\]
    	Dabei stellt $p_{(q^*,s)}$ den Anteil des Werts $s$ in dem $q^*$-Block dar.
   \end{block}
    
    \vfill
    
	\begin{block}{rekursive (c,l)-Diversität \tiny \cite{machanavajjhala_l_diversity}}
		Innerhalb eines $q^*$-Blocks sei $r_i$ die Anzahl des $i$-häufigsten sensiblen Attributs. Mit einer gegebenen Konstante $c$ erfüllt dieser $q^*$-Block \textbf{rekursive $(c, l)$-Diversität}, wenn $r_1 < c(r_l + r_{l+1} + ...  + r_m )$ gilt. Eine Tabelle $T^*$ erfüllt $(c, l)$-Diversität, wenn jeder $q^*$-Block $(c, l)$-Diversität erfüllt. $1$-Diversität ist immer erfüllt.
    \end{block}
\end{frame}


\begin{frame}{Beispiel: 2-diverse Tabelle}
	\begin{center}
		%Erstellt von script/generate_table.py - bei Bedarf anpassen
			\begin{tabular}{|l|c|c|r|l|}
		\hline \textit{Identifier} & \multicolumn{3}{c|}{\textbf{Nicht-sensibel}} & \textbf{Sensibel} \\ 
		\hline \textit{Name} & \textbf{Geschl.} & \textbf{PLZ} & \textbf{Geburtsjahr} & \textbf{Erkrankung} \\ \hline
		\hline \rowcolor{svshellblau1!30} 	- & w & 22*** & 1944-45 & Hepatitis \\ 
		\hline \rowcolor{svshellblau1!30}	- & w & 22*** & 1944-45 & Gicht \\ 
		\hline \rowcolor{svsgrau1!30} 		- & * & 22*** & 1949 & Arthrose \\ 
		\hline \rowcolor{svshellblau2!30} 	- & w & 228** & 1985-92 & Diabetes \\
		\hline \rowcolor{svshellblau2!30} 	- & w & 228** & 1985-92 & Demenz \\  
		\hline \rowcolor{white} 		- & m & 22997 & 1936-75 & Arthrose \\ 
		\hline \rowcolor{svsgrau1!30} 		- & * & 22*** & 1949 & Diabetes \\ 
		\hline \rowcolor{svsrot!30} 		- & m & 22*** & 1963-64 & Demenz \\ 
		\hline \rowcolor{svsrot!30} 		- & m & 22*** & 1963-64 & Demenz \\ 
		\hline \rowcolor{white} 		- & m & 22997 & 1936-75 & Hepatitis \\ 
		\hline 
		\end{tabular}
		\vspace{0.5cm}
        
		k-anonyme Tabelle mit \(k=2\), aber nur l-divers mit \(l=1\)
	\end{center}
\end{frame}


\begin{frame}{Beispiel: 2-diverse Tabelle}
	\begin{center}
		%Erstellt von script/generate_table.py - bei Bedarf anpassen
			\begin{tabular}{|l|c|c|r|l|}
		\hline \textit{Identifier} & \multicolumn{3}{c|}{\textbf{Nicht-sensibel}} & \textbf{Sensibel} \\ 
		\hline \textit{Name} & \textbf{Geschl.} & \textbf{PLZ} & \textbf{Geburtsjahr} & \textbf{Erkrankung} \\ \hline
		\hline \rowcolor{svshellblau1!30} 	- & w & 22*** & 1944-45 & Hepatitis \\ 
		\hline \rowcolor{svshellblau1!30}	- & w & 22*** & 1944-45 & Gicht \\ 
		\hline \rowcolor{svsgrau1!30} 		- & * & 22*** & 1949 & Arthrose \\ 
		\hline \rowcolor{svshellblau2!30} 	- & w & 228** & 1985-92 & Diabetes \\
		\hline \rowcolor{svshellblau2!30} 	- & w & 228** & 1985-92 & Demenz \\  
		\hline \rowcolor{svsrot!30} 		- & m & 22*** & 1936-75 & Arthrose \\ 
		\hline \rowcolor{svsgrau1!30} 		- & * & 22*** & 1949 & Diabetes \\ 
		\hline \rowcolor{svsrot!30} 		- & m & 22*** & 1936-75 & Demenz \\ 
		\hline \rowcolor{svsrot!30} 		- & m & 22*** & 1936-75 & Demenz \\ 
		\hline \rowcolor{svsrot!30} 		- & m & 22*** & 1936-75 & Hepatitis \\ 
		\hline 
		\end{tabular}
		\vspace{0.5cm}
        
		\textbf{Ergebnis:} k-anonyme Tabelle mit \(k=2\) und l-divers mit \(l=2\)
	\end{center}
\end{frame}


\subsection{Verbesserung zu k-Anonymität}

\begin{frame}{Verbesserung zu k-Anonymität}
	$l$-Diversität sichert verschiedene Ausprägungen der sensiblen Attribute in den verschiedenen \(q^*\)-Blöcken zu.
	
	\begin{itemize}
	\item Die \textit{Homogenity Attack} ist nicht mehr möglich.
	\item \textit{Background Knowledge Attacks} werden erschwert.
	\end{itemize}
	
	%Ein Vorteil von l-Diversity ist, dass vorhandene Algorithmen für k-Anonymität leicht angepasst werden können.
\end{frame}


\subsection{Schwächen der l-Diversität }

\begin{frame}{Schwächen der l-Diversität}
	\large \textcolor{gray!80}{Skewness attack} \normalsize \small\cite{Li2007t-closseness} 
	\begin{itemize}
		\item Tabelle mit einem sensiblen Attribut, 2 Ausprägungen.\\
	    \item Wahrscheinlichkeit für Ausprägung 1 ist sehr hoch.\\
	    \item Wahrscheinlichkeit für Ausprägung 2 entsprechend niedrig.\\
	    \item Es 2-diverse Tabelle mit Block $q^*$ vor.\\
	    \item $q^*$ beinhaltet zu 50\% Ausprägung 1 und zu 50\% Ausprägung 2.\\
	    \item Die Wahrscheinlichkeit, dass ein Tupel aus q* Ausprägung 2 besitzt, liegt nun bei 50\%.\\
	\end{itemize}
	
	\textbf{Beispiel:} Angenommen das sensible Attribut hat die Werte: krank / gesund. In der Bevölkerung sind 1\% krank und 99\% gesund. Die Wahrscheinlichkeit, dass eine Person aus dem Block $q^*$ krank ist liegt nun bei 50\% und nicht mehr bei 1\%.
\end{frame}

\begin{frame}{l-Diversität - Schwächen}
	\large \textcolor{gray!80}{Similarity attack} \normalsize \small\cite{Li2007t-closseness} 
	\vspace{0.2cm}

	l-Diversität garantiert, dass in jedem Block unterschiedliche sensible Werte stehen. Es kann jedoch vorkommen, dass sich diese Werte ähneln.

	\vspace{0.2cm}

	\textbf{Beispiel:} In einem Block stehen unterschiedliche Krankheiten als sensibles Attribut. Es sind nur Geschlechtskrankheiten.
	
	Wenn nun eine Person diesem Block zugeordnet werden kann, so weiß man auch, dass diese Person eine Geschlechtskrankheit hat.

	\vspace{0.2cm}
	\centering
	\begin{tabular}{|c|c|c|}
	\hline \textbf{G.jahr} & \textbf{PLZ} & \textbf{Erkrankung} \\
	\hline
     1970 & 21*** & Diabetes \\ 
	 1970 & 21*** & Syphilis \\ 
	 1970 & 21*** & Gicht \\
	\hline 
     1975 & 21*** & \cellcolor{svsrot}Tripper \\
     1975 & 21*** & \cellcolor{svsrot}Syphilis \\ 
	 1975 & 21*** & \cellcolor{svsrot}Chlamydien \\ 
	\hline 
	\end{tabular}

	\vspace{0.2cm}
	\(k = 3, l = 3\)
\end{frame}
