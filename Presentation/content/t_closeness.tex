\section{t-Closeness}

\begin{frame}{t-Closeness}
t-closeness stellt ein Maß für minimalen Wissensgewinn, der durch Betrachtung eines q*-Blocks im Vergleich zur gesamten Distribution entsteht dar \cite{kltHauf}.
\begin{block}{Definition t-closeness}
Eine Äquivalenzklasse(q*-Block) hat die Eigenschaft t-closeness, wenn die (semantische) Distanz zwischen der Verteilung der Werte eines sensitiven Attributes, innerhalb der Äquivalenzklasse, und der Verteilung der Werte des sensitiven Attributes, innerhalb der Tabelle, nicht größer als t ist. Eine Tabelle hat die Eigenschaft t-closeness, wenn diese Eigenschaft für alle Äquivalenzklassen gilt.	
\end{block}
\ \\
\ \\ 
Das Problem: Wie bestimmt man die (semantische) Distanz?
\end{frame}

\begin{frame}{Eearth Movers Distanz (EMD)}
Die Earth Movers Distanz (EMD)  basiert auf der minimalen Arbeit, die zu verrichten ist, um eine Distribution in eine andere umzuwandeln.
\begin{block}{Definition von EMD \cite{Li2007t-closseness}}
Gegeben sei $P=(p_1,...,p_m)$, $Q=(q_1,...,q_m)$, $d_{ij}$ ist die Grunddistanz zwischen $p_i$ und $q_j$. $f_{ij}$ ist die minimale Masse, die transportiert werden muss um $p_i$ in $q_i$ zu verwandeln. EMD ist dann die gesamte Arbeit die verrichtet werden muss $D[P,Q] = \sum_{i=1}^m \sum_{j=1}^m d_{ij} f_{ij}$.\\
Unter den folgenden Bedingungen
\begin{enumerate}[i)]
	\item $f_{ij}>0$ | $1\le i \le m, 1\le j \le m$		
	\item $p_i - \sum_{j=1}^{m}f_{ij} +\sum_{j=1}^{m}f_{ji} = q_i$ | $1 \le i \le m$
	\item $\sum_{i=1}^{m}\sum_{j=1}^{m} f_{ij} = \sum_{i=1}^{m} p_i = \sum_{i=1}^{m} q_i$ 
\end{enumerate}
\end{block}
\end{frame}

\begin{frame}{Eearth Movers Distanz (EMD)}
Aus den drei Bedingungen folgen die zwei Fakten \cite{Li2007t-closseness}:\\
\ \\
\textbf{Fakt 1:} If $\forall i,j | 0 \le d_{ij} < 1$ then $0 \le D[P,Q]  \le 1$. 
Das Bedeutet, dass wenn die Grunddistanz normalisiert ist, ist auch der EMD normalisiert. \textbf{Somit kann ein einheitliches Maß für t bestimmt werden.}\\
\ \\
\textbf{Fakt 2:} Gegeben sind zwei Äquivalenzklassen $E_1$ und $E_2$.
$P_1$ ist die Verteilung eines sensitiven Attributes aus $E_1$.
$P_2$ ist die Verteilung eines sensitiven Attributes aus $E_2$.
$P$ ist die Verteilung eines sensitiven Attributes aus $E_1 \cup E_2$. Dann gilt die folgende Ungleichung: \\
$D[P,Q] \le \frac{|E_1|}{|E_1|+|E_2|}D[P_1,Q] + \frac{|E_2|}{|E_1|+|E_2|}D[P_2,Q]$\\
$\Rightarrow D[P,Q] \le max(D[P_1,Q], D[P_2,Q])$ 
\end{frame}

\begin{frame}{Eearth Movers Distanz (EMD)}
\textbf{Fakt 2:} $D[P,Q] \le max(D[P_1,Q], D[P_2,Q])$ 
Dies Bedeutet, dass die maximale Distanz zwischen einer Äquivalenzklasse und der Tabelle beim zusammenführen zweier Äquivalenzklassen nicht steigt. \textbf{Somit bleibt die t-closeness Eigenschaft beim zusammenführen erhalten.} \\ 
\ \\
\textbf{Generalisation Property:} Sei $T$ eine Tabelle, $A$ und $B$ sind Generalisierungen von $T$, wobei$A$ mehr generalisiert ist als $B$. Wenn $B$ die Eigenschaft t-closeness hat, dann hat auch $A$ die Eigenschaft t-closeness.\\
\ \\
\textit{Beweis:} Die Äquivalenzklassen aus $A$ bestehen aus der vereinigung mehrerer Äquivalenzklassen aus $B$. Nach Fakt 2 kann somit maximale Distanz nicht größer werden. Somit hat auch $A$ die Eigenschaft t-closeness.  
\end{frame}

\begin{frame} {EMD Beispiel}

\begin{columns}[T]
  \begin{column}{0.5\textwidth}
    \begin{table}[]
      \centering
      \label{tclossenessExample}
      \begin{tabular}{|l|l|l|l}
      \hline
      \textbf{PLZ}   & \textbf{Alter}    & \textbf{Einkommen} \\\hline
      4767* & $\le 40$ & 3K \\
      4767* & $\le 40$ & 5K \\
      4767* & $\le 40$ & 9K \\\hline
      4790* & $\ge 40$ & 6K \\
      4790* & $\ge 40$ & 11K \\
      4790* & $\ge 40$ & 8K \\\hline
      4760* & $\le 40$ & 4K \\
      4760* & $\le 40$ & 7K \\
      4760* & $\le 40$ & 10K \\\hline
      \end{tabular}
      \caption{Einkommenstabelle}
    \end{table}
    \end{column}

      \begin{column}{0.5\textwidth}
        $Q = \{3k, 4k, 5k, 6k, 7k, 8k, 9k, 10k, 11k\}$
        
        $P_1 = \{3k, 4k, 5k\}$, and P2 = \(\{6k, 8k, 11k\}\). 
        
        We calculate
        D[P1,Q] and D[P2,Q]
    \end{column}
\end{columns}
\end{frame}

\section{Fazit}
\begin{frame} {Fazit}
k-Anonymität
	Anonymisiert Daten durch Generalisierung und Unterdrückung
    
l-Diversität
	mind l versch. sensible Werte pro Äquivalenzklasse
t-closseness
	- Verteilung  der Daten für kleinstmöglicher Wissensgewinn	
    - Herrausforderung: finden einer    	  Distanzfunktion
\end{frame}
