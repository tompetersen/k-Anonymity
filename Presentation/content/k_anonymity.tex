\section{k-Anonymität}

\begin{frame}{Beispiel: Private Tabelle}
	\begin{center}
		%Erstellt von script/generate_table.py - bei Bedarf anpassen
		\begin{tabular}{|l|c|c|r|l|}
		\hline \textbf{Identifikator} & \multicolumn{3}{c|}{\textbf{Nicht-sensibel}} & \textbf{Sensibel} \\ 
		\hline \textbf{Name} & \textbf{Geschl.} & \textbf{PLZ} & \textbf{Geb.dat.} & \textbf{Erkrankung} \\
		\hline \hline Sofia Müller & w & 22981 & 22.12.1944 & Hepatitis \\ 
		\hline Emma Weber & w & 22362 & 27.3.1945 & Gicht \\
		\hline Sofia Koch & w & 22669 & 3.9.1949 & Arthrose \\ 
		\hline Emilia Wagner & w & 22862 & 1.3.1985 & Diabetes \\ 
		\hline Emma Meyer & w & 22875 & 16.2.1992 & Demenz \\  
		\hline Noah Meyer & m & 22997 & 19.3.1936 & Arthrose \\ 
		\hline Elias Schäfer & m & 22121 & 26.11.1949 & Diabetes \\ 
		\hline Finn Fischer & m & 22350 & 28.11.1963 & Demenz \\ 
		\hline Leon Schmidt & m & 22188 & 26.4.1964 & Demenz \\ 
		\hline Elias Koch & m & 22997 & 7.10.1975 & Hepatitis \\ 
		\hline 
		\end{tabular}
	\end{center}
\end{frame}

\begin{frame}{Begriffe}
	\begin{description}
	\item[Expliziter Identifikator] Attribut, das ein Individuum (nahezu) eindeutig identifiziert. Beispiele: Name, Adresse, Steuernummer, ...
	
	\item[Sensibles Attribut] Attribut, dessen Wert für ein Individuum in einer Datenmenge nicht öffentlich gemacht werden soll.
	
	\item[Quasi-Identifikator] Eine Menge nicht-sensibler Attribute \(\{A_i, \dots, A_j\}\) einer Tabelle, deren Attribute mit einer externen Datenquelle verknüpft werden können, um mindestens ein Individuum der Gesamtmenge eindeutig zu identifizieren.
	\end{description}
\end{frame}

\begin{frame}{k-Anonymität}
	Eine Tabelle erfüllt \(k\)-Anonymität, wenn jede Zeile ununterscheidbar von mindestens \(k-1\) anderen Zeilen im Bezug auf einen Quasi-Identifikator ist.

	\vspace{0.5cm}

	\begin{block}{k-Anonymität} Sei \(T(A_1, \dots , A_n)\) eine Tabelle und \(Q_T=\{A_i, \dots, A_j\}\) der zugehörige Quasi-Identifikator.
	\vspace{0.1cm}

	\(T\) erfüllt \textbf{\(k\)-Anonymität} genau dann, wenn jede Belegung von Werten in \(T[Q_T]\) mindestens \(k\) mal auftritt, wobei \(T[Q_T]\) die duplikatenerhaltende Projektion von \(T\) auf die Attribute des Quasi-Identifikators beschreibt.\\
	Die so entstandenen Klassen werden auch als \textbf{\(q^*\)-Blöcke} bezeichnet.
	\end{block}

\end{frame}

\begin{frame}{Beispiel: k-anonyme Tabelle}
	\begin{center}
		%Erstellt von script/generate_table.py - bei Bedarf anpassen
		\begin{tabular}{|l|c|c|r|l|}
		\hline \textbf{Identifikator} & \multicolumn{3}{c|}{\textbf{Nicht-sensibel}} & \textbf{Sensibel} \\ 
		\hline \textbf{Name} & \textbf{Geschl.} & \textbf{PLZ} & \textbf{Geb.dat.} & \textbf{Erkrankung} \\
		\hline \hline Sofia Müller & w & 22981 & 22.12.1944 & Hepatitis \\ 
		\hline Emma Weber & w & 22362 & 27.3.1945 & Gicht \\
		\hline Sofia Koch & w & 22669 & 3.9.1949 & Arthrose \\ 
		\hline Emilia Wagner & w & 22862 & 1.3.1985 & Diabetes \\ 
		\hline Emma Meyer & w & 22875 & 16.2.1992 & Demenz \\  
		\hline Noah Meyer & m & 22997 & 19.3.1936 & Arthrose \\ 
		\hline Elias Schäfer & m & 22121 & 26.11.1949 & Diabetes \\ 
		\hline Finn Fischer & m & 22350 & 28.11.1963 & Demenz \\ 
		\hline Leon Schmidt & m & 22188 & 26.4.1964 & Demenz \\ 
		\hline Elias Koch & m & 22997 & 7.10.1975 & Hepatitis \\ 
		\hline 
		\end{tabular}
	\end{center}
\end{frame}

\begin{frame}{Beispiel: k-anonyme Tabelle}
	\begin{center}
		%Erstellt von script/generate_table.py - bei Bedarf anpassen
		\begin{tabular}{|l|c|c|r|l|}
		\hline \textit{Identifikator} & \multicolumn{3}{c|}{\textbf{Nicht-sensibel}} & \textbf{Sensibel} \\ 
		\hline \textit{Name} & \textbf{Geschl.} & \textbf{PLZ} & \textbf{Geb.dat.} & \textbf{Erkrankung} \\
		\hline \hline - & w & 22981 & 22.12.1944 & Hepatitis \\ 
		\hline - & w & 22362 & 27.3.1945 & Gicht \\
		\hline - & w & 22669 & 3.9.1949 & Arthrose \\ 
		\hline - & w & 22862 & 1.3.1985 & Diabetes \\ 
		\hline - & w & 22875 & 16.2.1992 & Demenz \\  
		\hline - & m & 22997 & 19.3.1936 & Arthrose \\ 
		\hline - & m & 22121 & 26.11.1949 & Diabetes \\ 
		\hline - & m & 22350 & 28.11.1963 & Demenz \\ 
		\hline - & m & 22188 & 26.4.1964 & Demenz \\ 
		\hline - & m & 22997 & 7.10.1975 & Hepatitis \\ 
		\hline 
		\end{tabular}
	\end{center}
\end{frame}

\begin{frame}{Beispiel: k-anonyme Tabelle}
	\begin{center}
		%Erstellt von script/generate_table.py - bei Bedarf anpassen
		\begin{tabular}{|l|c|c|r|l|}
		\hline \textit{Identifikator} & \multicolumn{3}{c|}{\textbf{Nicht-sensibel}} & \textbf{Sensibel} \\ 
		\hline \textit{Name} & \textbf{Geschl.} & \textbf{PLZ} & \textbf{Geburtsjahr} & \textbf{Erkrankung} \\ \hline
		\hline - & w & 22981 & 1944 & Hepatitis \\ 
		\hline - & w & 22362 & 1945 & Gicht \\
		\hline - & w & 22669 & 1949 & Arthrose \\ 
		\hline - & w & 22862 & 1985 & Diabetes \\ 
		\hline - & w & 22875 & 1992 & Demenz \\  
		\hline - & m & 22997 & 1936 & Arthrose \\ 
		\hline - & m & 22121 & 1949 & Diabetes \\ 
		\hline - & m & 22350 & 1963 & Demenz \\ 
		\hline - & m & 22188 & 1964 & Demenz \\ 
		\hline - & m & 22997 & 1975 & Hepatitis \\ 
		\hline 
		\end{tabular}
	\end{center}
\end{frame}

\begin{frame}{Beispiel: k-anonyme Tabelle}
	\begin{center}
		%Erstellt von script/generate_table.py - bei Bedarf anpassen
		\begin{tabular}{|l|c|c|r|l|}
		\hline \textit{Identifikator} & \multicolumn{3}{c|}{\textbf{Nicht-sensibel}} & \textbf{Sensibel} \\ 
		\hline \textit{Name} & \textbf{Geschl.} & \textbf{PLZ} & \textbf{Geburtsjahr} & \textbf{Erkrankung} \\ \hline
		\hline \rowcolor{svshellblau1!30} - & w & 22*** & 1944-45 & Hepatitis \\ 
		\hline \rowcolor{svshellblau1!30} - & w & 22*** & 1944-45 & Gicht \\
		\hline - & w & 22669 & 1949 & Arthrose \\ 
		\hline - & w & 22862 & 1985 & Diabetes \\ 
		\hline - & w & 22875 & 1992 & Demenz \\  
		\hline - & m & 22997 & 1936 & Arthrose \\ 
		\hline - & m & 22121 & 1949 & Diabetes \\ 
		\hline - & m & 22350 & 1963 & Demenz \\ 
		\hline - & m & 22188 & 1964 & Demenz \\ 
		\hline - & m & 22997 & 1975 & Hepatitis \\ 
		\hline 
		\end{tabular}
	\end{center}
\end{frame}

\begin{frame}{Beispiel: k-anonyme Tabelle}
	\begin{center}
		%Erstellt von script/generate_table.py - bei Bedarf anpassen
		\begin{tabular}{|l|c|c|r|l|}
		\hline \textit{Identifikator} & \multicolumn{3}{c|}{\textbf{Nicht-sensibel}} & \textbf{Sensibel} \\ 
		\hline \textit{Name} & \textbf{Geschl.} & \textbf{PLZ} & \textbf{Geburtsjahr} & \textbf{Erkrankung} \\ \hline
		\hline \rowcolor{svshellblau1!30} - & w & 22*** & 1944-45 & Hepatitis \\ 
		\hline \rowcolor{svshellblau1!30} - & w & 22*** & 1944-45 & Gicht \\
		\hline \rowcolor{svsgrau1!30} - & * & 22*** & 1949 & Arthrose \\ 
		\hline - & w & 22862 & 1985 & Diabetes \\ 
		\hline - & w & 22875 & 1992 & Demenz \\  
		\hline - & m & 22997 & 1936 & Arthrose \\ 
		\hline \rowcolor{svsgrau1!30}- & * & 22*** & 1949 & Diabetes \\ 
		\hline - & m & 22350 & 1963 & Demenz \\ 
		\hline - & m & 22188 & 1964 & Demenz \\ 
		\hline - & m & 22997 & 1975 & Hepatitis \\ 
		\hline 
		\end{tabular}
	\end{center}
\end{frame}

\begin{frame}{Beispiel: k-anonyme Tabelle}
	\begin{center}
		%Erstellt von script/generate_table.py - bei Bedarf anpassen
		\begin{tabular}{|l|c|c|r|l|}
		\hline \textit{Identifikator} & \multicolumn{3}{c|}{\textbf{Nicht-sensibel}} & \textbf{Sensibel} \\ 
		\hline \textit{Name} & \textbf{Geschl.} & \textbf{PLZ} & \textbf{Geburtsjahr} & \textbf{Erkrankung} \\ \hline
		\hline \rowcolor{svshellblau1!30} - & w & 22*** & 1944-45 & Hepatitis \\ 
		\hline \rowcolor{svshellblau1!30} - & w & 22*** & 1944-45 & Gicht \\
		\hline \rowcolor{svsgrau1!30} - & * & 22*** & 1949 & Arthrose \\ 
		\hline \rowcolor{svshellblau2!30} - & w & 228** & 1985-92 & Diabetes \\ 
		\hline \rowcolor{svshellblau2!30} - & w & 228** & 1985-92 & Demenz \\  
		\hline - & m & 22997 & 1936 & Arthrose \\ 
		\hline \rowcolor{svsgrau1!30}- & * & 22*** & 1949 & Diabetes \\ 
		\hline - & m & 22350 & 1963 & Demenz \\ 
		\hline - & m & 22188 & 1964 & Demenz \\ 
		\hline - & m & 22997 & 1975 & Hepatitis \\ 
		\hline 
		\end{tabular}
	\end{center}
\end{frame}

\begin{frame}{Beispiel: k-anonyme Tabelle}
	\begin{center}
		%Erstellt von script/generate_table.py - bei Bedarf anpassen
		\begin{tabular}{|l|c|c|r|l|}
		\hline \textit{Identifikator} & \multicolumn{3}{c|}{\textbf{Nicht-sensibel}} & \textbf{Sensibel} \\ 
		\hline \textit{Name} & \textbf{Geschl.} & \textbf{PLZ} & \textbf{Geburtsjahr} & \textbf{Erkrankung} \\ \hline
		\hline \rowcolor{svshellblau1!30} - & w & 22*** & 1944-45 & Hepatitis \\ 
		\hline \rowcolor{svshellblau1!30} - & w & 22*** & 1944-45 & Gicht \\
		\hline \rowcolor{svsgrau1!30} - & * & 22*** & 1949 & Arthrose \\ 
		\hline \rowcolor{svshellblau2!30} - & w & 228** & 1985-92 & Diabetes \\ 
		\hline \rowcolor{svshellblau2!30} - & w & 228** & 1985-92 & Demenz \\  
		\hline - & m & 22997 & 1936 & Arthrose \\ 
		\hline \rowcolor{svsgrau1!30}- & * & 22*** & 1949 & Diabetes \\ 
		\hline \rowcolor{svsrot!30} - & m & 22*** & 1963-64 & Demenz \\ 
		\hline \rowcolor{svsrot!30} - & m & 22*** & 1964-64 & Demenz \\ 
		\hline - & m & 22997 & 1975 & Hepatitis \\ 
		\hline 
		\end{tabular}
	\end{center}
\end{frame}

\begin{frame}{Beispiel: k-anonyme Tabelle}
	\begin{center}
		%Erstellt von script/generate_table.py - bei Bedarf anpassen
			\begin{tabular}{|l|c|c|r|l|}
		\hline \textit{Identifikator} & \multicolumn{3}{c|}{\textbf{Nicht-sensibel}} & \textbf{Sensibel} \\ 
		\hline \textit{Name} & \textbf{Geschl.} & \textbf{PLZ} & \textbf{Geburtsjahr} & \textbf{Erkrankung} \\ \hline
		\hline \rowcolor{svshellblau1!30} 	- & w & 22*** & 1944-45 & Hepatitis \\ 
		\hline \rowcolor{svshellblau1!30}	- & w & 22*** & 1944-45 & Gicht \\ 
		\hline \rowcolor{svsgrau1!30} 		- & * & 22*** & 1949 & Arthrose \\ 
		\hline \rowcolor{svshellblau2!30} 	- & w & 228** & 1985-92 & Diabetes \\
		\hline \rowcolor{svshellblau2!30} 	- & w & 228** & 1985-92 & Demenz \\  
		\hline \rowcolor{white} 			- & m & 22997 & 1936-75 & Arthrose \\ 
		\hline \rowcolor{svsgrau1!30} 		- & * & 22*** & 1949 & Diabetes \\ 
		\hline \rowcolor{svsrot!30} 		- & m & 22*** & 1963-64 & Demenz \\ 
		\hline \rowcolor{svsrot!30} 		- & m & 22*** & 1963-64 & Demenz \\ 
		\hline \rowcolor{white} 			- & m & 22997 & 1936-75 & Hepatitis \\ 
		\hline 
		\end{tabular}
		\vspace{0.5cm}
		\pause

		\textbf{Ergebnis: }k-anonyme Tabelle mit \(k=2\)
	\end{center}
\end{frame}


\subsection{Generalisierung}

\begin{frame}{Generalisierung}
	%	\textbf{s. Samarati, Sweeney Kapitel 3}
	%	
	%	domain, ground domain, generalization (partial ordering on domains)
	%	
	%	generalized table
	\begin{block} {Generalisierung}
		Vergröberung der Werte, die ein Attribut annehmen kann (Generalisierung auf Attributebene).	
	\end{block}
	
	\vspace{0.5cm}
	
	Beispiele für Generalisierungshierarchien: 
	\begin{enumerate}
	\item PLZ: \\
		\hspace{0.4cm} \(P_0 = \{22765, 22769, 22529, 20246\}\) \begin{small}\textit{Grundwertebereich}\end{small} \\
		\(\rightarrow P_1 = \{2276^*, 2252^*, 2024^*\}\)\\
		\(\rightarrow P_2 = \{2^{****}\}\)\\
	\item Geschlecht: \\
		\hspace{0.4cm} \(G_0 = \{\text{männlich, weiblich}\}\) \begin{small}\textit{Grundwertebereich}\end{small} \\
		\(\rightarrow G_1 = \{\text{nicht\_veröffentlicht}\}\)
	\end{enumerate}
\end{frame}


\begin{frame}{Generalisierung II}
	
%	Domain generalization, exampletables, s. Sweeney/Samarati p. 7
	\begin{columns}[T]
		\begin{column}{0.3\textwidth}
			\begin{tikzpicture}
			    \node[] (g0p0) {$G_0P_0$};
			    \node[] (g1p0) [above left = 0.5cm and 0.2cm of g0p0]  {$G_1P_0$};
			    \node[] (g1p1) [above = 0.5cm of g1p0] {$G_1P_1$};
				\node[] (g0p1) [above right = 0.5cm and 0.2cm of g0p0] {$G_0P_1$};
				\node[] (g0p2) [above = 0.5cm of g0p1] {$G_0P_2$};
				\node[] (g1p2) [above left = 0.5cm and 0.2cm of g0p2] {$G_1P_2$};
			
			    \path[draw,thick,->]
			    (g0p0) edge node {} (g0p1)
			    (g0p0) edge node {} (g1p0)
			    (g0p1) edge node {} (g0p2)
				(g0p1) edge node {} (g1p1)
				(g1p0) edge node {} (g1p1)
				(g1p1) edge node {} (g1p2)
				(g0p2) edge node {} (g1p2);
			\end{tikzpicture}
			
			\small Generalisierungshierarchie für Attributmenge 
			\vspace{0.2cm}

			\tiny Jeder Pfad von \(G_0P_0\) zu \(G_1P_2\) stellt einen möglichen Weg der Generalisierung dar.
		\end{column}
	
		\only<1>{
			\begin{column}{0.1\textwidth}
			\end{column}
			\begin{column}{0.6\textwidth}
				PLZ: \\
				\hspace{0.4cm} \(P_0 = \{22765, 22769, 22529, 20246\}\)\\
				\(\rightarrow P_1 = \{2276^*, 2252^*, 2024^*\}\)\\
				\(\rightarrow P_2 = \{2^{****}\}\)\\
				Geschlecht: \\
				\hspace{0.4cm} \(G_0 = \{\text{männlich, weiblich}\}\) \\
				\(\rightarrow G_1 = \{\text{nicht\_veröffentlicht}\}\)
			\end{column}
		}

		\only<2>{
			\begin{column}{0.65\textwidth}
				\centering		
				\(T_{G_0P_0}\)
		
				\vspace{0.5cm}
				\small
				\begin{tabular}{|c|c|}
				\hline \textbf{Geschlecht} & \textbf{PLZ} \\
				\hline m & 22765 \\ 
				\hline m & 22765 \\ 
				\hline m & 22769 \\ 
				\hline m & 22529 \\ 
				\hline m & 20246 \\ 
				\hline w & 22765 \\ 
				\hline w & 22765 \\ 
				\hline w & 22769 \\ 
				\hline w & 22529 \\ 
				\hline w & 22529 \\ 
				\hline w & 22529 \\ 
				\hline w & 20246 \\ 
				\hline 
				\end{tabular}
			\end{column}
		}
		
		\only<3>{
			\begin{column}{0.65\textwidth}
				\centering		
				\(T_{G_1P_0}\)
		
				\vspace{0.5cm}
				\small
				\begin{tabular}{|c|c|}
				\hline \textbf{Geschlecht} & \textbf{PLZ} \\
				\hline * & 22765 \\ 
				\hline * & 22765 \\ 
				\hline * & 22769 \\ 
				\hline * & 22529 \\ 
				\hline * & 20246 \\ 
				\hline * & 22765 \\ 
				\hline * & 22765 \\ 
				\hline * & 22769 \\ 
				\hline * & 22529 \\ 
				\hline * & 22529 \\ 
				\hline * & 22529 \\ 
				\hline * & 20246 \\ 
				\hline 
				\end{tabular}
			\end{column}
		}

		\only<4>{
			\begin{column}{0.65\textwidth}
				\centering		
				\(T_{G_0P_1}\)
		
				\vspace{0.5cm}
				\small
				\begin{tabular}{|c|c|}
				\hline \textbf{Geschlecht} & \textbf{PLZ} \\
				\hline m & 2276* \\ 
				\hline m & 2276* \\ 
				\hline m & 2276* \\ 
				\hline m & 2252* \\ 
				\hline m & 2024* \\ 
				\hline w & 2276* \\ 
				\hline w & 2276* \\ 
				\hline w & 2276* \\ 
				\hline w & 2252* \\ 
				\hline w & 2252* \\ 
				\hline w & 2252* \\ 
				\hline w & 2024* \\ 
				\hline 
				\end{tabular}
			\end{column}
		}

		\only<5>{
			\begin{column}{0.65\textwidth}
				\centering		
				\(T_{G_1P_2}\)
		
				\vspace{0.5cm}
				\small
				\begin{tabular}{|c|c|}
				\hline \textbf{Geschlecht} & \textbf{PLZ} \\
				\hline * & 2**** \\ 
				\hline * & 2**** \\ 
				\hline * & 2**** \\ 
				\hline * & 2**** \\ 
				\hline * & 2**** \\ 
				\hline * & 2**** \\ 
				\hline * & 2**** \\ 
				\hline * & 2**** \\ 
				\hline * & 2**** \\ 
				\hline * & 2**** \\ 
				\hline * & 2**** \\ 
				\hline * & 2**** \\ 
				\hline 
				\end{tabular}
			\end{column}
		}
	\end{columns}
\end{frame}


\begin{frame}{Generalisierung III}
%	k-minimale Generalisierung
	\textbf{Aber}: Nicht jede Generalisierung ist gleichermaßen sinnvoll!
	\pause
	
%	Triviallösung: Für jedes Attribut die höchste Stufe der Generalisierung wählen -> Jedes Tupel bezogen auf den Quasi identifier enthält die gleichen Werte -> auf Kosten hoher Generalisierung und damit geringer Nutzbarkeit der Daten.
%	
	\begin{block}{k-minimale Generalisierung.}
		\(T_i\) ist die \(k\)-minimale Generalisierung einer Tabelle \(T\) gdw.
		\begin{itemize}
			\item \(T_i\) \(k\)-Anonymität erfüllt und
			\item keine Tabelle \(T_j\) existiert, die ebenfalls \(k\)-Anonymität erfüllt und für die \(T_i\) eine Generalisierung darstellt.
		\end{itemize}
	\end{block}
\end{frame}


\subsection{Suppression}

\begin{frame}{Unterdrückung}
%	Suppression auf Tupelebene
%	
%	in combination with generalization (Abwägen, treshold, ...)
	\begin{block}{Unterdrückung}
		Entfernen von Daten aus der Tabelle - hier auf Tupelebene, d.h. Tupel können nur komplett entfernt werden.\\Unterdrückung ist jedoch auch auf Attributebene möglich (entspricht dann maximaler Generalisierung).
	\end{block}

	\begin{columns}[T]
		\begin{column}{0.3\textwidth}
			\centering
			\begin{tabular}{|c|c|}
				\hline \textbf{G.} & \textbf{PLZ} \\
				\hline m & 22765 \\ 
				\hline w & 22765 \\ 
				\hline m & 22769 \\ 
				\hline w & 22769 \\ 
				\hline m & 80043 \\
				\hline
			\end{tabular}
			\vspace{0.3cm}

			Daten
		\end{column}
	
		\begin{column}{0.3\textwidth}
			\centering			
			\begin{tabular}{|c|c|}
				\hline \textbf{G.} & \textbf{PLZ} \\
				\hline m & * \\ 
				\hline w & * \\ 
				\hline m & * \\ 
				\hline w & * \\ 
				\hline m & * \\
				\hline
			\end{tabular}
			\vspace{0.3cm}

			Generalisierung
		\end{column}

		\begin{column}{0.3\textwidth}
			\centering
			\begin{tabular}{|c|c|}
				\hline \textbf{G.} & \textbf{PLZ} \\
				\hline m & 2276* \\ 
				\hline w & 2276* \\ 
				\hline m & 2276* \\ 
				\hline w & 2276* \\ 
				\hline
			\end{tabular}
			\vspace{0.6cm}

			Unterdrückung\\ \& Generalisierung
		\end{column}
	\end{columns}
\end{frame}

\begin{frame}[c]{Implementierungen}
	\begin{center}
		Die Berechnung von k-anonymen Tabelle ist NP-schwer, ...\\
		\pause
		\vspace{0.5cm}

		... es wurden jedoch \(\mathcal{O}(k)\)-Approximationsalgorithmen gefunden \cite{aggarwal, meyerson}.
	\end{center}
\end{frame}
	
\begin{frame}{Implementierungen}
	\begin{center}
		\begin{tabular}{|c||c|c|c|c|}
			\hline & \multicolumn{4}{c|}{\textbf{Unterdrückung}}\\
			\hline \textbf{Generalisierung} & Tupel & Attribut & Zelle & Keine \\ \hline
			\hline Attribut & \textbf{AG\_TS} & AG\_AS \tiny = AG & \textbf{AG\_CS} & AG \tiny = AG\_AS \\ 
			\hline Zelle & \sout{CG\_TS} & \sout{CG\_AS} & CG\_CS \tiny = CG & CG \tiny = CG\_CS\\ 
			\hline Keine & TS & AS & CS & - \\ 
			\hline 
		\end{tabular} 
		\vspace{0.2cm}
	
		\tiny Klassifizierung von Techniken für die Erstellung k-anonymer Tabellen. Entnommen aus \cite{Ciriani}
	\end{center}

	\pause

	\begin{columns}[c]
		\begin{column}{0.5\textwidth}	
			\hfill \textbf{Implementationen}
		\end{column}
	
		\begin{column}{0.5\textwidth}
			\begin{itemize}
				\item \(\mu\)-Argus
				\item Datafly
				\item Incognito
				\item Mondrian
				\item ...
			\end{itemize}
		\end{column}
	\end{columns}
\end{frame}


\subsection{Schwächen der k-Anonymität}

\begin{frame}{Schwächen der k-Anonymität}
	\begin{itemize}
		\item \textit{Complementary release attack}: Veröffentlichung mehrerer \(k-\)anonymer Tabellen unterschiedlicher Generalisierung kann bei Kombination dieser Tabellen die \(k-\)Anonymität verletzen \cite{sweeney_k_anonymity}. %p.11
	
		\item \textit{Temporal attack}: Dynamische Tabellen können \(k-\)Anonymität verletzen \cite{sweeney_k_anonymity}. %p.12

		\item \textbf{Unsorted matching attack} \cite{sweeney_k_anonymity} % Veröffentlichung mehrerer \(k-\)anonymer Tabellen mit derselben Sortierung ausgehend von einer nicht-öffentlichen Tabelle \cite{sweeney_k_anonymity}. %p.10

		\item \textbf{Homogeneity attack} \cite{machanavajjhala_l_diversity} %Gleichheit der sensitive attributes einer Gruppe, die sich in den Werten des quasi identifiers gleicht, leakt das sensitive attribute eines Individuums \cite{machanavajjhala_l_diversity}. %p. 2
	
		\item \textbf{Background knowledge attack} \cite{machanavajjhala_l_diversity} %Nutzen von Hintergrundwissen, um mit hoher Wahrscheinlichkeit auf den Wert des sensitive attributes eines Individuumsin einer Gruppe zu schließen \cite{machanavajjhala_l_diversity}. %p.2/4
	\end{itemize}
	%	\begin{tabular}{|c|c|c|}
	%	\hline \textbf{G.jahr} & \textbf{PLZ} & \textbf{Erkrankung} \\
	%	\hline 1970 & 21985 & Hepatitis O \\ 
	%	\hline 1970 & 21986 & Hepatitis P \\ 
	%	\hline 1970 & 21724 & Hepatitis Q \\ 
	%	\hline 1970 & 21725 & Hepatitis R \\ 
	%	\hline 1975 & 21985 & Hepatitis S \\ 
	%	\hline 1975 & 21986 & Hepatitis T \\ 
	%	\hline 1975 & 21724 & Hepatitis U \\ 
	%	\hline 1975 & 21725 & Hepatitis V \\ 
	%	\hline 1980 & 21985 & Hepatitis W \\ 
	%	\hline 1980 & 21986 & Hepatitis X \\ 
	%	\hline 1980 & 21724 & Hepatitis Y \\ 
	%	\hline 1980 & 21725 & Hepatitis Z \\ 
	%	\hline 
	%	\end{tabular}
\end{frame}


\begin{frame} {Unsorted matching attack}
	Veröffentlichung mehrerer \(k\)-anonymer Tabellen mit derselben Sortierung ausgehend von einer nicht-öffentlichen Tabelle  identifiziert Individuen.
	\begin{columns}[T] 
	     \begin{column}[T]{0.5\textwidth} 
			\centering
			\footnotesize
			\begin{tabular}{|c|c|}
			\hline \textbf{G.jahr} & \textbf{PLZ} \\
			\hline 1970-80 & 21985 \\ 
			\hline 1970-80 & 21986 \\ 
			\hline 1970-80 & 21724 \\ 
			\hline 1970-80 & 21725 \\ 
			\hline \rowcolor{svsgrau1}1970-80 & 21985 \\ 
			\hline 1970-80 & 21986 \\ 
			\hline 1970-80 & 21724 \\ 
			\hline 1970-80 & 21725 \\ 
			\hline 1970-80 & 21985 \\ 
			\hline 1970-80 & 21986 \\ 
			\hline 1970-80 & 21724 \\ 
			\hline 1970-80 & 21725 \\ 
			\hline 
			\end{tabular}
			\vspace{0.3cm}

			\(k = 3\)
	     \end{column}
	
	     \begin{column}[T]{0.5\textwidth} 
			\centering
			\footnotesize
			\begin{tabular}{|c|c|c|}
			\hline \textbf{G.jahr} & \textbf{PLZ} & \textbf{Erkrankung} \\
			\hline 1970 & 2198* & Hepatitis X \\ 
			 1970 & 2198* & Hepatitis Y \\ 
			\hline 1970 & 2172* & Hepatitis Z \\ 
			 1970 & 2172* & Hepatitis X \\ 
			\hline \rowcolor{svsgrau1} 1975 & 2198* & Hepatitis Y \\ 
			 1975 & 2198* & Hepatitis Z \\ 
			\hline 1975 & 2172* & Hepatitis X \\ 
			 1975 & 2172* & Hepatitis Y \\ 
			\hline 1980 & 2198* & Hepatitis Z \\ 
			 1980 & 2198* & Hepatitis X \\ 
			\hline 1980 & 2172* & Hepatitis Y \\ 
			 1980 & 2172* & Hepatitis Z \\ 
			\hline 
			\end{tabular}
			\vspace{0.3cm}

			\(k = 2\)
	     \end{column}
     \end{columns}
	
%	\vspace{0.5cm}
%	\centering
%	Zufällige Sortierung der Tabellen verhindert diesen Angriff!
\end{frame}


%\begin{frame} {Complementary release attack}
%	\textbf{TBD}
%\end{frame}
%
%
%\begin{frame} {Temporal attack}
%	\textbf{TBD?}
%\end{frame}


\begin{frame} {Homogeneity attack}
	Gleichheit der sensiblen Attribute einer Gruppe, die sich in den Werten des Quasi-Identifikatoren gleicht, verrät das sensible Attribut eines Individuums.
	\vspace{0.5cm}	

	\centering
	\small
	\begin{tabular}{|c|c|c|}
	\hline \textbf{G.jahr} & \textbf{PLZ} & \textbf{Erkrankung} \\
	\hline 1970 & 21*** & Hepatitis X \\ 
	 1970 & 21*** & Hepatitis Y \\ 
	 1970 & 21*** & Hepatitis Z \\ 
	 1970 & 21*** & Hepatitis Y \\ 
	\hline 1975 & 21*** & \cellcolor{svsrot}Hepatitis X \\ 
	 1975 & 21*** & \cellcolor{svsrot}Hepatitis X \\ 
	 1975 & 21*** & \cellcolor{svsrot}Hepatitis X \\ 
	 1975 & 21*** & \cellcolor{svsrot}Hepatitis X \\
	\hline 
	\end{tabular}
	\vspace{0.3cm}

	\(k = 4\)
\end{frame}


\begin{frame} {Background knowledge attack}
	Nutzen von Hintergrundwissen, um mit hoher Wahrscheinlichkeit auf den Wert des sensiblen Attributes eines Individuums in einer Gruppe zu schließen.
	\vspace{0.5cm}

	\centering
	\small
	\begin{tabular}{|c|c|c|}
	\hline \textbf{G.jahr} & \textbf{PLZ} & \textbf{Erkrankung} \\
	\hline 1970 & 21*** & Hepatitis X \\ 
	 1970 & 21*** & Hepatitis Y \\ 
	 1970 & 21*** & Hepatitis Z \\ 
	 1970 & 21*** & Hepatitis Y \\ 
	\hline 1975 & 21*** & \cellcolor{svshellblau1}Hepatitis X \\ 
	 1975 & 21*** & \cellcolor{svshellblau1} Hepatitis X \\ 
	 1975 & 21*** & \cellcolor{svsrot}Hepatitis Y \\ 
	 1975 & 21*** & \cellcolor{svsrot}Hepatitis Y \\
	\hline 
	\end{tabular}
	\vspace{0.3cm}

	\(k = 4\)
	\vspace{0.3cm}

	\textbf{Hintergrundwissen}: Hepatitis X tritt nur bzw. mit hoher Wahrscheinlichkeit lediglich bei Männern auf.
\end{frame}
